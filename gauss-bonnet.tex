\documentclass[oneside,a4paper]{amsart}
%\usepackage{maa-monthly}
\usepackage{gauss-bonnet}
%\usepackage[russian,english]{babel}
%\usepackage[utf8]{inputenc}

\begin{document}

\title{An exercise on the comparison theorem}
\author{Anton Petrunin and Sergio Zamora Barrera}
%\address{Anton Petrunin,  Math. Dept., PSU, University Park,  PA 16802, USA.}
%\address{aqp6@psu.edu}
%\address{Sergio Zamora Barrera,  Math. Dept., PSU,University Park,  PA 16802, USA.}
%\address{sxz38@psu.edu}

\keywords{discrete minimal surface, polyhedral surface, area minimizing surface, minimal surface.}
\maketitle

The following exotic problem was used it as an advanced exercise in our textbook~\cite{petrunin-zamora};
it requires creativity in applying comparison theorems.

{

\begin{wrapfigure}{r}{30 mm}
\vskip-4mm
\centering
\includegraphics{mppics/pic-1}
\bigskip
\includegraphics{mppics/pic-3}
\end{wrapfigure}

\begin{thm}{Problem}
Suppose $\Sigma$ is a smooth closed convex in the Euclidean space. 
Show that a closed geodesic $\gamma$ on $\Sigma$ cannot have a configuration of self-intersections as shown on the first diagram.

In other words, show that $\gamma$ cannot cut $\Sigma$ into one quadrangle, one pentagon, and three monogons.
\end{thm}

If you limit number of self-intersections to 3, then the given diagram is the only one that requires substantial work.

For example, the problem for the second diagram is much simpler.
It can appear as a closed geodesic on a convex surface;
an example can be constructed by fattening an equilateral triangle.
However, the answer is ``no'' if one assumes that the surface is \emph{strictly} convex. 
It follows from the Gauss--Bonnet formula.
We recommend to solve the second problem before reading further.

}

\section*{Comparison theorem}

Further we assume that $\Sigma$ is a closed convex surface;
in particular, $\Sigma$ is homeomorphic to the sphere and has nonnegative Gauss curvature.

Suppose that $\Delta$ is an $n$-gon on $\Sigma$ with geodesic sides.
From the \emph{Gauss--Bonnet formula} it follows that sum of the internal angles of $\Delta$ cannot be smaller than $(n-2)\cdot\pi$.

In particular, if $\Delta$ is a triangle with angles $\alpha$, $\beta$, and $\gamma$, then
\[\alpha+\beta+\gamma\ge\pi.\leqno({*})\]

If we assume in addition that the sides of $\Delta$ are length minimizing geodesics among the curves in $\Delta$ with the same endpoints, then the inequality $({*})$ can be made more exact.

Namely consider the so-called \emph{model triangle} $\tilde\Delta$ of $\Delta$; that is, $\tilde\Delta$ is a plane triangle with equal corresponding sides.
Since the sides are length-minimizing in $\Delta$, they satisfy the triangle inequality; therefore the model triangle is defined.

Denote bu $\tilde \alpha$, $\tilde \beta$ and $\tilde \gamma$ the angles of $\tilde\Delta$ respectively.
Then 
\[
\alpha> \tilde \alpha,
\qquad
\beta> \tilde \beta,
\qquad
\text{and}
\qquad
\gamma> \tilde \gamma.
\leqno({*}{*})
\]
Since $\tilde\alpha+\tilde\beta+\tilde\gamma=\pi$, this inequality implies $({*})$.

The inequality $({*}{*})$ easily follows from the proof of the \emph{Toponogov comparison theorem}.
We leave it as an exercise to those who knows this proof;
the rest of the readers should simply believe that it is true.

\section*{Proof}

Arguing by contradiction, suppose that $\Sigma$ contains such a geodesic $\gamma$;
assume that arcs and angles are labeled as on the left diagram.

\begin{figure}[!ht]
\begin{minipage}{.38\textwidth}
\centering
\includegraphics{mppics/pic-472}
\end{minipage}\hfill
\begin{minipage}{.58\textwidth}
\centering
\includegraphics{mppics/pic-473}
\end{minipage}
\end{figure}


Applying the Gauss--Bonnet formula to the quadrangle and pentagon that the geodesic cuts from $\Sigma$, we get that
\[2\cdot\alpha\le\beta+\gamma
\qquad\text{and} \qquad
2\cdot\beta+2\cdot \gamma\le\pi+\alpha.\leqno(\asterism)\]
It follows that $\alpha \le\tfrac \pi 3$.


Consider the part of the geodesic $\gamma$ without the arc~$a$.
It cuts from $\Sigma$ a pentagon $\Delta$ with sides and angles shown on the diagram on the right.

\begin{wrapfigure}{r}{50 mm}
\vskip-0mm
\centering
\includegraphics{mppics/pic-474}
\vskip8mm
\end{wrapfigure}

Since small segments of any geodesic on $\Sigma$ are length minimizing, we can add additional vertices of the sides of $\Delta$ so that each side becomes length-minimizing.
Choose a vertex $v$ and subdivide $\Delta$ into triangles by joining $v$ to other vertices of the broken geodesic.

Consider a model triangle for each triangle in the subdivision so that they shared sides as in $\Delta$.
By comparison inequality $({*}{*})$, the angles of the model triangles do not exceed the corresponding angles of the original triangle.
Therefore the model triangles form a plane pentagon $\tilde\Delta$ with convex polygonal sides.
Moreover its angles do not exceed the corresponding angles of $\Delta$.

\begin{wrapfigure}{r}{50 mm}
\vskip-4mm
\centering
\includegraphics{mppics/pic-500}
\vskip0mm
\end{wrapfigure}

It remains to show described plane polygogon does not exist.
Let us orient its  sides counterclockwise;
denote the obtained vectors by $s_1,\dots,s_k$.

On the other hand, the conditions on the angles of polygons imply that the vectors $s_i$ point in the complement of white sectors shown with angles marked on the diagram.
The sum of magnitudes of the vectors in each black sector is also marked.
Let $v$, $v'$, $w$, $w'$ be the marked unit vectors;
set $r=v+v'+w+w'$.
Observe that $(\asterism)$ implies that $r\ne 0$,
and, moreover, 
\[\langle r,s_1\rangle+\dots+\langle r,s_k\rangle>0,\]
but  $s_1+\dots+s_k=0$ --- a contradiction.


{\sloppy
\printbibliography[heading=bibintoc]
\fussy
}

\end{document}
