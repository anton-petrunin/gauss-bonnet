\documentclass[oneside,a4paper]{amsart}

\usepackage{gauss-bonnet}
\hypersetup{
pdftitle={Self-crossing geodesics},
pdfauthor={Anton Petrunin}
}

\begin{document}
%\pagestyle{empty}\renewcommand\includegraphics[2][{}]{}


\title{Self-crossing geodesics}
\author{Anton Petrunin}
\maketitle

\section{Introduction}


The rubber band on the picture is pulled around a pebble,
and it crosses itself at several points.
\begin{figure}[!ht]
\hfill
\begin{minipage}{.56\textwidth}
\centering
\includegraphics[width=\textwidth]{pics/pebble.jpg}
\end{minipage}
\hfill
\begin{minipage}{.30\textwidth}
\centering
\includegraphics{mppics/pic-50}
\end{minipage}
\hfill
\end{figure}
The combinatorics of self-crossings can be described by a closed plane curve --- it is the rubber band in a parametrization of the surface with one point removed.
For example, if you could turn the pebble around you would see that the self-crossings are described by the plane curve on the right diagram.

We assume that the surface of the pebble is strongly convex, smooth, and frictionless;
in this case, the rubber band models a closed geodesic.
Suppose that we are interested in possible patterns of self-crossings; more precisely:

\medskip

\emph{What are the possible combinatoric types of self-crossings of a closed geodesic on a strongly convex smooth closed surface?}

\begin{figure}[ht!]
\begin{center}
\includegraphics{mppics/pic-55}
\end{center}
\end{figure}

Consider the six possible patterns with three double crossings.
Configurations 1, 2, 3, and 4 can be realized as mirror-symmetric geodesics on mirror-symmetric surfaces; the projections on the plane of symmetry are sketched.
\begin{figure}[ht!]
\begin{center}
\includegraphics{mppics/pic-100}
\end{center}
\end{figure}

Further, we will discuss \emph{forbidden} configurations;
that is, configurations that cannot appear for closed geodesic.
These are configurations 5 and 6.

This question is a good exercise --- it could be explained to anyone, but an answer requires a considerable part of the theory.
The reader is welcome to check that there are no forbidden patterns with less than 3 double crossings and
try the cases with more self-crossings (check \cite[Figures 15--17]{arnold}).
By the way, \emph{is there an algorithm for solving the general case?} 
Our question is closely related to the so-called \emph{flat knot types} of geodesics;
see \cite{angenent} and the references therein.

In what follows, we discuss the Gauss--Bonnet formula as well as the Alexandrov--Toponogov theorem and apply them to forbid configurations 5 and 6.
These theorems are covered in our textbook \cite{petrunin-zamora} which I like, altho they are treated in plenty of other places as well.

\section*{Gauss--Bonnet and no 5}

Suppose that $\Delta$ is an $n$-gon with geodesic sides in a surface $\Sigma$.
Recall that by the \emph{Gauss--Bonnet formula} the sum of the external angles of $\Delta$ equals
\[2\cdot\pi\cdot\chi(\Delta)-\int\limits_\Delta K,\]
where $\chi(\Delta)$ denotes the \emph{Euler characteristic} of $\Delta$ and $K$, the Gauss curvature of~$\Sigma$.

Further, we assume that $\Sigma$ is a closed strongly convex surface. In this case,
\begin{itemize}
 \item $\Sigma$ has strictly positive Gauss curvature;
 \item $\Sigma$ is homeomorphic to the sphere and therefore $\chi(\Sigma)=2$;
 \item %by Jordan--Schönflies theorem, 
 $\Delta$ is homeomorphic to the disc and therefore $\chi(\Delta)=1$.
\end{itemize}
It follows that the sum of the internal angles of $\Delta$ is lager than $(n\z-2)\cdot\pi$.
In particular, if $\Delta$ is a triangle with angles $\alpha$, $\beta$, and $\gamma$, then
\[\alpha+\beta+\gamma>\pi.\leqno({*})\]
The Gauss--Bonnet formula can be applied to the whole surface; it implies that
the integral of Gauss curvature along  $\Sigma$ is exactly $4\cdot\pi$.

{

\begin{wrapfigure}{r}{33 mm}
\vskip-8mm
\centering
\includegraphics{mppics/pic-3}
\end{wrapfigure}

\parit{No 5 is forbidden.}
Suppose there is a geodesic with self-crossings as on the diagram;
it divides the surface $\Sigma$ into one triangle, say $\Delta$, one hexagon, and three monogons.
Denote by $\alpha$, $\beta$, and $\gamma$ the internal angles of $\Delta$.

Note that three monogons have internal angles $\alpha$, $\beta$, and~$\gamma$.
By Gauss--Bonnet, the integral of Gauss curvature along each monogon is 
$\pi\z+\alpha$, $\pi\z+\beta$, and $\pi\z+\gamma$ 
respectively.
By $({*})$ the integral of Gauss curvature along the three monogons exceeds $4\cdot \pi$.
But $4\cdot \pi$ is the integral of Gauss curvature along the \emph{whole} surface  --- a contradiction.

}

\section*{Alexandrov--Toponogov and no 6}

Let $\Delta$ be a geodesic triangle with angles $\alpha$, $\beta$, and $\gamma$ on the surface $\Sigma$.
Assume that the sides of $\Delta$ are length-minimizing among the curves \emph{in} $\Delta$ with the same endpoints, then the inequality $({*})$ can be made more exact.

Namely consider the so-called \emph{model triangle} $\tilde\Delta$ of $\Delta$; that is, $\tilde\Delta$ is a plane triangle with equal corresponding sides.
Since the sides are length-minimizing, they satisfy the triangle inequality; therefore the model triangle is defined.

Denote by $\tilde \alpha$, $\tilde \beta$ and $\tilde \gamma$ the angles of $\tilde\Delta$ respectively.
Then 
\[
\alpha> \tilde \alpha,
\qquad
\beta> \tilde \beta,
\qquad
\text{and}
\qquad
\gamma> \tilde \gamma.
\leqno({*}{*})
\]
Since $\tilde\alpha+\tilde\beta+\tilde\gamma=\pi$, this inequality implies $({*})$.

The inequality $({*}{*})$ easily follows from the proof of the \emph{Alexandrov--Toponogov theorem}.
The latter implies that $({*}{*})$ holds for triangles with length-minimizing sides in the \emph{whole} surface.
The proof is left as an 
exercise for those familiar with the Alexandrov--Toponogov theorem; others may simply accept it as true.

\parit{No 6 is forbidden.}
Suppose that such a geodesic $\xi$ exists;
assume that its arcs and angles are labeled as in the leftmost part of the diagram below.
Applying the Gauss--Bonnet formula to the quadrangle and pentagon that $\xi$ cuts from the surface, we get that
\[2\cdot\alpha<\beta+\gamma,
 \quad
2\cdot\beta+2\cdot \gamma<\pi+\alpha,
\quad\text{and therefore} \quad \alpha <\tfrac \pi 3.\leqno(\asterism)\]


\begin{figure}[!ht]
\vskip-1mm
\begin{minipage}{.22\textwidth}
\centering
\includegraphics{mppics/pic-472}
\end{minipage}
\hfill
\begin{minipage}{.35\textwidth}
\centering
\includegraphics{mppics/pic-473}
\end{minipage}
\hfill
\begin{minipage}{.35\textwidth}
\centering
\includegraphics{mppics/pic-474}
\end{minipage}
\vskip-1mm
\end{figure}

Consider the part of $\xi$ without the arc labeled by~$a$.
It cuts from the surface a pentagon $\Delta$ with sides and angles shown in the middle part of the diagram.

Let us add additional vertices on the sides of $\Delta$ so that each side becomes length-minimizing in $\Delta$.
Choose a vertex of $\Delta$ and join it by shortest paths in $\Delta$ to every other vertex.
Consider a model triangle for each triangle in the obtained subdivision of $\Delta$;
the model triangles lie in the plane and we suppose that they share sides as in $\Delta$.
By the comparison inequality $({*}{*})$, the angles of the model triangles do not exceed the corresponding angles of the original triangle.
Therefore the model triangles form a convex plane polygon, say $\tilde\Delta$,
such that
\begin{itemize}
\item The five angles of $\tilde\Delta$ that correspond to the angles of $\Delta$ do not exceed those.
\item Each side of $\tilde\Delta$ equals to the corresponding small side of $\Delta$.
\end{itemize}
It remains to show that no convex plane polygon meets these two conditions.

\begin{wrapfigure}{r}{47 mm}
\vskip-2mm
\centering
\includegraphics{mppics/pic-475}
\bigskip
\includegraphics{mppics/pic-500}
\vskip7mm
\end{wrapfigure}

Let us orient the sides $\tilde\Delta$ counterclockwise;
denote the obtained vectors by $s_1,\dots,s_k$.
Note that the vectors $s_i$ point in the complement of white sectors shown below with angles marked.
The sum of the magnitudes of the vectors in each black sector is also marked.

By $(\asterism)$ we can choose a vector $r$ as on the diagram, so that $\phi\z>\tfrac{\pi-\beta}2$ and $\psi>\tfrac{\pi-\gamma}2$.
Note that $(\asterism)$ implies inequalities on inner products
\[\langle r,v\rangle+\langle r,v'\rangle<0\quad\text{and}\quad \langle r,w\rangle+\langle r,w'\rangle<0\qquad\qquad\qquad\qquad\qquad\qquad\qquad\]
for any unit vectors $v$, $v'$, $w$, and $w'$ in the marked black sectors.
It follows that 
\[
\langle r,s_1\rangle+\dots+\langle r,s_k\rangle<0.\qquad\qquad\qquad\qquad\qquad\qquad
\]
On the other hand, the vectors $s_i$ circumambulate $\tilde\Delta$;
so, the sum has to vanish
--- a contradiction.

{\small \parbf{Acknowledgments.}
I want to thank Arseniy Akopyan, Maxim Arnold, Serge Tabachnikov, David Kramer, and Sergio Zamora Barrera for their help.
This work was partially supported by the NSF grant DMS-2005279, and Simons Foundation grant no 584781.}



{\sloppy
\printbibliography
\fussy
}

\end{document}
